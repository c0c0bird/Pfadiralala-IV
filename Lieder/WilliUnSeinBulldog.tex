\beginsong{Willi un sein Bulldog}[
    wuw={Fäägmeel}, 
    index={De Willi un sein Bulldog}, 
]

\beginverse
\[D]De Willi un sein Bulldog sei bei jedem Werrer drauß,
dem \[A]Willi un seim Bulldog, mecht \[D]das naut aus!
Wann‘s \[D]hachlt und gewirrert und kaan Mensch will aus em Haus,
dann \[A]geht äuser Willi erst richtich aus sich \[D]raus.
\endverse

\beginchorus
Er hot en ^Rärrer-Kerre-Buldogg, un en Rärrer-Kerre-Wa
un en ^Rärrer-Kerre-Kärrnche, fir de ^Goarte un die Fraa.
Er hot en ^Rärrer-Kerre-Buldogg, un en Rärrer-Kerre-Wa
er ^fiert bei jedem Werrer, was will man do noch ^sa?
\endchorus

\beginverse
Wann‘s im ^Fruijohr richtich nass is un es raat wochelang 
un ^jeder schennt weil kaaner uff sei ^Äcker kann; 
dann ^hert un seiht mer trotzdem of em Feld erim en Mann, 
weil ^der ja mit seim Bulldogg bei jedem Wärrer ^kann.
\endverse

\printchorus

\beginverse
Im ^Winter wenn huch Schnee lait un es es richtich kaalt, 
mächt der ^Willi mit seim Buldogg ^Holz im Waald. 
Un ^fällt 'emol de Schneepflug weil er stecke bleibt aus, 
kimmt de ^Willi merrem Buldogg un zeiht en wirrer ^raus.
\endverse

\printchorus

\beginchorus
^Rärrer-Kerre–Buldogg (Rärrer-Kerre–Buldogg),
^Rärrer-Kerre–Buldogg (Rärrer-Kerre–Buldogg),
^Rärrer-Kerre–Buldogg, ^Rärrer-Kerre-Wa, 
^Rärrer-Kerre Kärrnche, was will man do noch ^sa?
\endchorus


\endsong 
