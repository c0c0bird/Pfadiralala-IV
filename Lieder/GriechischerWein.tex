\beginsong{Griechischer Wein}[
    mel={Udo Jürgens}, 
    txt={Michael Kunze}, 
    jahr={1974}, 
    index={Es war schon dunkel}, 
    tonart={Cm},
]

\beginverse
\endverse
\includegraphics[draft=false, page=1]{Noten/GriechischerWein.pdf}

\beginverse
Es war schon \[Cm]dunkel, als ich durch Vorstadtstraßen \[E]heimwärts ging.
Da war ein Wirtshaus, aus dem das Licht noch auf den \[Bb]Gehsteig schien.
Ich hatte \[Cm]Zeit und mir war \[G7]kalt, drum trat ich \[Cm]ein.

Da saßen \[Cm]Männer mit braunen Augen und mit \[E]schwarzem Haar,
und aus der Jukebox erklang Musik, die fremd und \[Bb]südlich war.
Als man m\[Cm]ich sah, stand \[G7]einer auf und lud mich \[Cm]ein.
\endverse

\beginchorus
\[A]Griechischer Wein ist so wie das Blut der Erde.
\[E]Komm', schenk dir ein, und wenn ich dann traurig werde,
\[Bb]liegt es daran, dass ich immer träume von \[G]daheim... Du musst ver\[G7]zeih'n.

\[A]Griechischer Wein, und die altvertrauten Lieder.
\[E]Schenk' noch mal ein, denn ich fühl' die Sehnsucht wieder,
\[Bb]in dieser Stadt werd' ich immer nur ein Fremder \[Em]sein\[H7], und al\[Em]lein.
\endchorus

\beginverse
Und dann ^erzählten sie mir von grünen Hügeln, ^Meer und Wind,
von alten Häusern und jungen Frauen, die al^leine sind,
und von dem ^Kind, das seinen ^Vater noch nie ^sah.

Sie sagten ^sich immer wieder: Irgendwann geht ^es zurück.
Und das Ersparte genügt zu Hause für ein ^kleines Glück.
Und bald denkt ^keiner mehr da^ran, wie es hier ^war.
\endverse

\printchorus

\endsong
